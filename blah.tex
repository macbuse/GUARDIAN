
The Pope has said that transexual Spanish should not use 
whatever I say will always be wrong

There may be no other and what could I say then



The Vatican has told a Spanish bishop that transsexuals cannot be godparents after he asked for a formal answer on the matter, the cleric in the diocese of Cadiz and Ceuta said.

Bishop Rafael Zornoza Boy said the Vatican’s doctrinal arm replied that transsexuals “publicly show an attitude contrary to the moral requirement to resolve one’s sexual identity problem according to the truth of one’s sex”.

In a statement on his diocese’s website, Zornoza Boy said: “Pope Francis has effectively said on several occasions, in line with church teaching, that this behaviour is against man’s nature.”

The church “wants to help everyone in their own situation with a compassionate heart, but without denying the truth it preaches,” the bishop added.

Zornoza Boy said he made the query to the congregation for the doctrine of the faith because of confusion among the faithful, and the publicity given to the complicated subject.

An Italian conservative Catholic blog, Rorate Caeli, said the issue was first raised by a transsexual, born a woman, who had asked to be godfather to his nephew.

In its response to Zornoza Boy, the Vatican said its stance on transsexual godparents “should not be seen as discrimination”.
NHS England is to be taken to court by the UK’s equality watchdog for failing to offer fertility services to transgender patients.

The Equality and Human Rights Commission will launch a high-profile judicial review action, a legal manoeuvre that is likely to prove controversial at a time when the NHS is struggling to balance budgets and provide core services.

Last month the Observer reported that the commission had written to NHS England putting it on notice that it needed to offer fertility services to transgender patients before they underwent treatment for gender dysphoria, a process that normally results in a loss of fertility.

By extracting and storing eggs and sperm before they undergo gender reassignment, transgender people can choose to have their biological children delivered via surrogates later in life. However, many are keen to proceed with treatment when they are teenagers, and may not have the resources to pay for such fertility services. This, the commission argues, discriminates against transgender people, whom it says should be offered the services as a standard procedure.

NHS England maintains that it is not its responsibility to ensure that fertility treatment is available to all patients, including transgender people. Currently it falls to individual clinical commissioning groups – the bodies that buy services for patients – to decide whether to provide them on the NHS, but many choose not to do so for transgender patients, according to the commission.

After receiving NHS England’s response , the commission has said it will now start legal proceedings. It is understood that NHS England continues to argue that it is under no obligation to offer the service at a national level.

“We have received a comprehensive response from NHS England to our letter regarding the provision of fertility services for transgender patients before they undergo treatment for gender dysphoria,” said Rebecca Hilsenrath, chief executive of the commission.

“We are proceeding with our judicial review claim and will remain in discussions with NHS England about the need to ensure the transgender community can access health services free from discrimination, and that individuals do not have to choose between treatment for gender dysphoria and the chance to start a family.”

The commission argues that gender dysphoria falls within NHS England’s specialised commissioning remit, which ensures appropriate treatment is given to those with complex conditions.

The legal case is likely to shine a light on how a large increase in people requesting transitioning treatment is placing greater demands on the health service.

Penny Mordaunt, minister for women and equalities, has launched an inquiry into what has driven a 4,400% increase in the number of girls being referred for transitioning treatment in the past decade. In 2009, 40 girls were referred by doctors for gender treatment. Last year the number had risen to 1,806.

The role of social media and the teaching of transgender issues in schools will form part of the inquiry.

A spokeswoman for NHS England said: “NHS England has responded in detail to the EHRC explaining why we believe their request is both misjudged and potentially unfair to NHS patients. If, however, they still decide to sue the NHS, the courts will consider the matter in the usual way.”
Within 12 hours of Bruce Jenner’s double coming out on network television Friday night – in which the former Olympian told ABC News that he identifies both as a woman and as a Republican – the hostility he is likely to face from some of his newly revealed party associates was on full display in Minnesota.

On Saturday morning, the Twittersphere was still digesting Jenner’s comments to Diane Sawyer that “for all [intents and purposes], I am a woman” and that he has always been “more on the conservative side”. But at the same moment, Republican leaders were gathering on the floor of Minnesota’s house of representatives to promote a vicious legislative attack on transgender rights.

They were pushing a new provision, HF 1546, that would ban students from using school bathrooms, showers and changing rooms other than those for the gender into which they were assigned.

Similar moves have swept Republican-held assemblies in six other US states this year – Florida, Kentucky, Massachusetts, Missouri, Nevada and Texas. Opponents of the bills call them “bathroom bounty laws” because in some cases they allow individuals to sue for up to $4,000 should they find themselves sharing bathroom facilities with a transgender person.

Though none of the bills has passed so far, they are seen as a sign of a new aggressiveness on the part of state-level Republicans in publicly expressing overt hostility towards transgender people. As such, they highlight some of the political and ideological tensions that Jenner could now face as a conservative embarking on gender transition.

Mara Keisling, executive director of the National Center for Transgender Equality and a trans woman herself, said that she saw no essential contradiction between Jenner’s transgender identity and his conservatism if by that he meant a belief in efficient and small government.

“It’s not Republicans who think government should get out of people’s lives who are the problem, it’s those who want to put it right in our faces,” she said.

When Jenner said he was “kind of more on the conservative side”, Sawyer asked him: “Are you Republican?”

“Yeah,” he said. “Is that a bad thing? I believe in the constitution.”

Keisling said she would welcome Jenner’s help in fighting what she called “mean-spirited, time-wasting knuckleheads” within the Republican party who were promoting the new bathroom legislation. “We welcome anybody who’s prepared to get involved in dealing with this legal nonsense.”

In his Sawyer interview, Jenner – who has asked that until further notice he still be referred to as “he” – said he was willing to ask the Republican leaders in Congress, John Boehner and Mitch McConnell, to engage with trans issues. He said he expected them to be “very receptive to it”.

But the relationship between Republicans and the LGBT community has come under renewed stress recently as the large pool of potential candidates vying to become the party’s presidential nominee in 2016 have been tying themselves in knots over same-sex marriage.

The tension burst to the surface on Sunday when a gay hotelier apologized in the face of an LGBT boycott for his “poor judgment” in hosting a political event by Republican senator Ted Cruz, who has been touting his opposition to gay marriage around the country.

The main LGBT group on the right, Log Cabin Republicans, has been swift in inviting Jenner to join them, seeing in the former athlete and reality TV star a potential champion of their cause. The organization’s national executive director, Gregory Angelo, said that the idea that transgender identity and conservatism were mutually exclusive was a myth put out through the media by the gay left.

“There’s a very diverse LGBT community out there that aligns itself with basic conservative principles,” he said.

Angelo added that the philosophy of the group was to engage fellow Republicans with the 90% of politics in which they were in agreement, and then take on the more difficult 10%. “We remind the GOP of the roots of the party – in equality, in emancipation, suffrage and civil rights.”

Not everyone is convinced that accommodation is possible, however. Jimmy LaSalvia, an openly gay conservative strategist and founder of the now defunct GoProud, said that he decided last year to quit the Republican party after a decade of trying to reform it from within.

“The reason I left was that the Republican party made it clear that there was no place for LGBT communities within it. There’s just not the support on the right for that kind of diversity,” he said.

LaSalvia said that there were plenty of LGBT people attracted to conservative economic policies. “But more and more LGBT Americans are finding it embarrassing to commit themselves as Republicans even if they are conservatives.”

LaSalvia believes that Jenner’s coming out will make it all the more difficult for Republicans to stick to the past. “The interview was so powerful because millions of Americans look up to him. So now when Republicans push policies that are openly hostile towards transgender people, what they are doing will hit home for everyone.”
Transgender rights activists and supporters demonstrate in front of the White House after the Trump administration revoked Obama-era guidance stating federal law requires transgender students to have unfettered access to bathrooms and locker rooms matching their gender identity
A news website aimed at British schoolchildren has agreed to pay an unsubstantiated amount after it implied that JK Rowling’s comments on gender caused harm to trans people.

The Day, which is recommended by the Department for Education and is designed to prompt teenagers to discuss current affairs, faced legal action from the Harry Potter author after publishing an article entitled: “Potterheads cancel Rowling after trans tweet”.

In the article, which some schools issued as homework, children were told that Rowling had objected to the use of the expression “people who menstruate” in place of “women”. It also referenced objections to Rowling’s recent comments from Harry Potter actors such as Daniel Radcliffe.

The original article in the Day asked teenagers to consider whether it is possible still to enjoy great works of art by “deeply unpleasant people” such as Pablo Picasso and Richard Wagner.

It said: “Since the 1950s, the civil rights movement has used boycotts to take money and status away from people and organisations harming minorities and shame them into change [sic] their behaviour. Online it is often called ‘cancelling’.”

The Day, which was founded and is run by the former Daily Express editor Richard Addis and is sold through subscriptions to around 1,500 schools, has now apologised after Rowling hired libel lawyers. The Day said: “We accept that our article implied that what JK Rowling had tweeted was objectionable and that she had attacked and harmed trans people. The article was critical of JK Rowling personally and suggested that our readers should boycott her work and shame her into changing her behaviour. Our intention was to provoke debate on a complex topic.

“We did not intend to suggest that JK Rowling was transphobic or that she should be boycotted. We accept that our comparisons of JK Rowling to people such as Picasso, who celebrated sexual violence, and Wagner, who was praised by the Nazis for his antisemitic and racist views, were clumsy, offensive and wrong.

“Debate about a complex issue where there is a range of legitimate views should have been handled with much more sensitivity and more obvious recognition of the difference between fact and opinion. We unreservedly apologise to JK Rowling for the offence caused, are happy to retract these false allegations and to set the record straight. We shall be making a financial contribution to a charity of JK Rowling’s choice.”

Rowling, 54, recently set out her views in a lengthy essay, in which she revealed that she had been a victim of domestic abuse, while saying she objected to proposals in Scotland to simplify the process by which transgender people can change the sex on their birth certificates.

Rowling’s comments on gender were condemned by LGBT charities and the leading stars of her Harry Potter film franchise. Last month the Guardian revealed that four authors at Rowling’s literary agency had resigned in protest after the company refused to issue a public statement of support for transgender rights, saying that “freedom of speech can only be upheld if the structural inequalities that hinder equal opportunities for underrepresented groups are challenged and changed”.

Rowling and hundreds of other writers, including Margaret Atwood, Salman Rushdie and Martin Amis, signed a letter published in Harper’s magazine criticising “an intolerance of opposing views, a vogue for public shaming and ostracism” online.

The letter said: “The free exchange of information and ideas, the lifeblood of a liberal society, is daily becoming more constricted.”

Despite calls for a boycott of Rowling’s works, her publisher, Bloomsbury, said that sales of her Harry Potter books had increased during the lockdown as parents bought copies to read with their children during the pandemic.
Donald Trump’s ban on transgender people serving in the US military has spread a pall of fear over the 15,000 personnel it touches and emboldened hostility towards even those on the frontline of active duty, an investigation by the Guardian has found.

The ban kicked in on 12 April and, two months into its imposition, the full chilling effects are only now becoming apparent. Under its terms, trans people seeking to enlist in the military are subject to an almost total exclusion unless they keep their gender identity hidden.

Most individuals who are already serving are similarly forced to keep their true selves tightly closeted. There are exceptions, but they are narrowly defined and hard to procure.

Last week, Trump tried to justify the ban by complaining about the high costs associated with treating military personnel for gender dysphoria, the formal diagnosis when an individual’s gender is different from the one assigned to them at birth. In fact the entire medical budget for gender transition-related care is a tenth what the armed forces spend annually on Viagra and Cialis.

The US president has also blamed trans individuals for causing “tremendous disruption” within the armed forces. That conflicts with the experience of 19 countries including Australia, Canada, Germany and the UK that have allowed trans people to serve without incident.

In the US, all four military service chiefs have testified before Congress that there were no known negative effects during the three years in which President Obama opened the doors to trans people.

“The biggest impact of the ban is that we are denying ourselves future heroes. Our nation needs the best and finest to fight and win future wars and we are turning away people just because they are trans,” said Lt Col B Fram, communications director of SPART*A, an education and advocacy group representing trans service members.

The Guardian partnered with SPART*A to investigate how Trump’s ban is bearing down on trans women and men in active duty settings. Here we profile four people in the navy and air force.

All are in aircrew and at the sharp end of the US fighting machine. They are familiar with the intense sacrifices to family and self that a military career involves, and have put their lives on the line in conflict zones.

Yet they are now having to cope with severe pressures brought about by the ban. That includes mounting hostility from transphobic peers who see Trump’s move as license to taunt and ridicule, as well as the daily fear that if they are outed as trans they could lose everything.

The stakes are now so high that all four spoke to the Guardian insisting on absolute anonymity. As one of them put it: “If I were found out by even one person, that would be the end of my flying career.”

When you meet Emily Finnerty in person, as the Guardian did recently, she comes across as more Tom Cruise than Tom Cruise. She has the same piercing gaze and verbal intensity of the Top Gun star, especially when describing the sensation of piloting an F-18 Super Hornet, the fighter jet that Cruise will fly in the sequel to the classic movie scheduled for release next year.

If Cruise’s role in Top Gun is Hollywood’s attempt to personify American military might and patriotic service, then Finnerty is the real deal. She knows what it’s like to fly at Mach speeds in that state-of-the-art $90m fighting machine. She has felt the punch in the gut when the aircraft explodes from the deck of a US navy carrier and endured the bone-crushing sustained 7.5gs of the dogfights that followed.

She is familiar with the terror and exhilaration of going zero to 160mph in two seconds. She can recall the glory of flying low through the canyons of Utah or the beauty of hugging the mountains of California. And she’s been there when training gets serious, notching up 60 combat missions in the all-too-real war zones of Iraq and Afghanistan.

Through all that, there has been the sacrifice that being on the frontlines of the US military extracts from her family and from herself. “My children didn’t know me, I was away so much. My back’s destroyed from the G-force. I’ve had near-death experiences from aircraft failure and I’ve seen friends die. But I’ve never betrayed the trust of my country and I’ve always answered the call.”

The US military has spent more than $11m in turning Finnerty from a young navy recruit over a decade ago into a lieutenant commander today. But now it is in danger of squandering every cent of that investment, by effectively valuing her as worthless.

Since the age of 10, Finnerty has secretly struggled with dysphoria. Though assigned male at birth, and presenting in public and within the navy as male, she has strong urges to transition and live authentically as a woman. 

 She came out to her wife and parents only a few weeks ago and is working through the trauma of that shattering revelation with her supportive family. If it is hard at home, it is far more difficult at work. Under the terms of Trump’s ban she will only be able to stay in the navy if she continues to wrap herself in a lie. She must put aside any desire to live as her authentic self and forego any medical treatment in order to present as a man and be allowed to continue to serve.

She is puzzled by the vast contrast between the faith that the military has put in her over so many years, and the utter lack of faith it is showing her now. “They have invested millions in me, and I have performed my duties to an exceptional level,” she said. “Yet because I see myself as a woman, I’m somehow less worthy to serve.”

The ironies of Finnerty’s position are not lost on her. Her mission within the military, she believes, is to serve the US constitution and uphold the freedom and human rights of her fellow Americans and people around the world. Yet to do so, she must forego her own freedom and human rights.

“My fellow service members and I have tried to build a better, safer world for people, but it’s a world I can’t live in,” she said.

She is being told that it is not in the best interests of the US military to have trans individuals openly serving. Yet for years she has been at the forefront of the most powerful fighting force on Earth, a closeted trans individual, earning plaudits and promotions all the way. To prove it she shows us her white dress jacket dripping with medals.

She could earn much more money and have an easier life for her family as a civilian – there are no prohibitions against flying as a commercial airline pilot. Yet she knows that were she to ask her superiors to reciprocate such personal sacrifice with even the most basic help, such as a session with a counselor about her dysphoria, that could put her wings and her entire Top Gun career in jeopardy.

“What do you think that does for the health of the individual?” she said. “Honestly, it’s a mess. I can’t be authentic and I’m not allowed to do anything about it.”

Finnerty is too much of an officer and a gentlewoman to criticize Trump for the trans ban. All she will say is: “I believe he thinks he’s doing what’s right. He’s the commander-in-chief.”

But she does confess to being disappointed and distressed by the order. “I’m angry, mostly because I think as a country we can do better. We are the frickin’ United States of America, we were founded on freedom. But here we are saying, ‘You can have your freedom, unless …’ That doesn’t jibe with me.”

What Trump and his joint chiefs of staff need to hear loud and clear is that Finnerty is now questioning her future in the services. At a time of a stunning shortage in fighter pilots, she is contemplating making an exit.

“It has made me think about getting out,” she said. “Maybe it’s time to go and be authentic.”

M, an active duty lieutenant and winged aviator in the navy, is most at risk of being evicted from the military of any of the four service members who spoke to the Guardian.

The cause of his precarious position is that he has begun to transition to male – without informing his superiors. As he puts it: “I am walking the fine line of fighting for myself but still fighting for my country, which, for trans individuals, is an absurdly immense challenge.”

He has had chest-contouring surgery as the first major step towards medically transitioning as a trans man. Now he knows that unless he succeeds in keeping his true existence utterly secret, his young, successful career in the navy could be snuffed out.

As the ban states, “transgender persons who require or have undergone gender transition are disqualified”. In short: get discovered, and you’re out.

The Guardian’s interaction with him was conducted in tightest security. He asked only to be referred to as “M” – the abbreviation of his male name that is in itself known only to very few. He also declined to discuss certain aspects of his story, such as the shocking harassment he endured from a former skipper, on grounds that some of his aircrew might recognize the detail and draw conclusions.

Over six years of military service, M has learned how to keep his head down and out of the spotlight. Despite two full tours of service fighting Isis in Iraq and Syria, many military awards and a spotless record, he has been forced to embrace secrecy.

“The ban has made me very careful around my squadron and fellow aircrew, including my wardroom. I’m exceedingly careful not to engage in discussion about the ban, or politics in general, and I keep my relationships with my peers strictly professional. I’ve established myself as being ‘very quiet’, and my silence is now considered normal.”

Despite his male identity, others continue to know him as the “female officer with the short haircut”. He grits his teeth when at formal functions he’s forced to wear women’s dress uniform with its antiquated 17th-century pants with side buttons, distinct suit top and hat that bears no resemblance to the men’s classic wide-brimmed sailor’s hat.

“It makes me feel like a fraud. I feel less and less like I’m participating in the great traditions of the navy, and more like I’m performing in costume. I’m not the same man who just a month ago was in a flight suit like everyone else flying combat missions.”

There’s another aspect of the double life M now lives under the ban. To those around him he is a demure, softly spoken, understated woman. Internally, he is a profoundly hurt and saddened man.

In emails to the Guardian, he laments that were he to openly begin testosterone hormone replacement therapy he would be downed from aviation permanently. He describes as “absurd” and “stone-aged” the fact that US armed forces still classify gender dysphoria as a “mental disorder”.

He is convinced that the most advantageous policy for the military would be the exact opposite of Trump’s ban. Allowing M to transition would actually benefit the navy because he could fulfill his potential in service to the nation. “I think my service would be even better if I didn’t have the anxiety of keeping closeted hanging over me,” he said.

At the root of it all, he thinks, is the ignorance and fear of the American public towards trans people. “So often, transgender individuals are made out to be these ‘scary’ concepts in faraway cities that you hear about on the news but have never met in real life. People – both civilian and military – need to realize that we are people too and make huge sacrifices that they would never dream of.”

The mood has darkened in Jennifer’s command since the ban came in. A master sergeant and flight engineer in the air force, she has noticed that leadership in the higher ranks have started to pursue witch-hunts seeking to detect and remove trans individuals from the service.

At lower levels, the climate has grown even uglier. “Those who were against us have become far more outspoken about how we don’t belong,” she said. “It’s as though in this extremely transphobic environment they feel they have to bash trans military members merely to fit in.”

US air force flying operations have always been among the most hostile corners of the military for trans people. Even during the brief three-year period under Obama when restrictions were generally loosened, being outed in the air force would still cost aircrew their wings.

That prohibition is toughened under the new ban, with the added burden that verbal abuse surrounding it has intensified. When someone has the temerity to come out as trans in the air force, Jennifer said, news spreads quickly, they are promptly removed from flying status and they are branded in mess-room banter as “non-mission hackers” and “losers”.

“It doesn’t matter if they have an exemplary past record or that they are devoted to flying and deploying. None of that positive information gets out. All that is discussed is that they are trans and they don’t belong.”

Jennifer knows that the same brutal logic applies to her. As a trans woman who is required to be scrupulously male on base, it doesn’t matter that she has had an exemplary career in the air force spanning more than 20 years. It doesn’t matter that she’s flown more than 6,000 hours being qualified on several different aircraft and has been honored with more than 20 ribbons and medals.

“If I were found out by even one person, that would be the end of my flying career,” she said.

So, like M, she is very strict about security and keeping her womanhood secret. Since the ban kicked in, she said, “I have become far more guarded and watch everything I do, from my mannerisms to managing when I exercise and shower, where I go off-base after work and how I dress. I cannot ever let my guard down.”

She avoids alcohol for fear that it might lull her into disclosing personal information. She wears baggy clothes to hide her physique. She preplans every move she makes and thinks constantly that the sacred trust of those around her, the trust upon which the US military is founded, would instantly be eviscerated if anybody discovered her true self.

And yet her devotion to the air force remains constant. At times, though, she admitted, “my drive does take a hit when I am willing to die for freedoms others enjoy but I am being denied”.

She added: “I love my country and it’s people, even the ones who think I should be banned. I believe if they just had a chance to meet one of us and realize we are normal people trying to live our life, doing our best every day, they might change their mind.”

Caleb is a petty officer who serves as a naval aircrewman, operating equipment in tactical missions around the world. He is the only one of the four service members profiled here to take advantage of a little known caveat within Trump’s trans ban known as the “grandfather clause”.

It allows people who were already in the armed forces and who acquired a diagnosis of gender dysphoria before the ban kicked in on 12 April to carry on serving in their correct gender. In other words, if you could get a military doctor officially to record your trans status in time, you could avoid the prohibition.

The clause sparked a frantic scramble by Caleb and many like him to secure a diagnosis by the deadline. Assigned female at birth, he is desperate to transition and is keen to start hormone replacement therapy and surgery.

But he is also desperate to hang on to his flight status and with it the valued role he so dearly loves that has taken him on numerous combat missions to Syria and other conflict zones. As he put it to the Guardian from his current deployment overseas: “It may be cheesy but every day when I walk on the flightline my pride surges. Often I look out the window over places I never dreamed I’d see.”

Applying for a diagnosis against the clock was hair-raising in itself. There was a long waiting list to see the one doctor at the behavioral health unit in his command who treats trans patients.

As the deadline approached, he took the hazardous step of visiting the general medical clinic to lobby staff for a rapid diagnosis. That was risky as it involved coming out as a trans man to people he did not know and could not necessarily trust.

Clinic workers were suspicious that he was seeking diagnosis as a ruse to shirk deployment. That really irked him, as the truth was precisely the reverse: he was delaying transition and potentially harming his own health so that he could carry on flying tactical missions.

By the time he finally got to see the doctor and have a diagnosis written into his record, it was with only days to spare. He squeaked through, along with up to 2,000 other trans service members who SPART*A estimates have been grandfathered in.

You might think that qualifying for the clause, and with it an exemption from the ban, would be the end of Caleb’s troubles. Not so. Though he has formally been recognized as a trans person, he feels far too unsettled by what has happened to go ahead with transition.

“On paper I’m safe from the ban, but I’m nervous that if I come out now I might lose my chance to fly. It would only take one person in my chain of command who doesn’t agree with transgender people being in the military for me to potentially be taken off deployments.”

To add to the jitters, the Pentagon made clear in the memo setting out the ban that the grandfather clause is legally tentative. The memo explicitly threatens that were the clause ever used to challenge the ban in court, then it would automatically be scrapped and everyone who had enjoyed its protection would immediately be expelled.

That leaves Caleb in uncharted territory. “I am proceeding with caution and being careful about coming out,” he said.

He remains closeted to all but a very few. He secretly binds his chest as one small concession to his authentic male identity, but wears his hair longer than he would choose to avoid attracting attention.

While he is delaying taking hormones to start medical transition because he fears it might lead to him being taken off flight status, he is grateful that being grandfathered in at least gives him the choice eventually to be openly trans while still remaining in the military. He says that makes him feel lucky.

He was also pleasantly surprised that when he privately came out to two co-workers recently they were generous and supportive. That has given him hope that one day he might be liberated without utterly being ostracized by those around him.

For now, though, he remains trapped in limbo. “I’m having to choose between serving my country in the job I love and being myself, and it’s really tough.”
Police officers unlawfully interfered with a man’s right to freedom of expression by turning up at his place of work to speak to him about allegedly “transphobic” tweets, the high court has ruled.

Harry Miller, a former police officer who founded the campaign group Fair Cop, said the actions of Humberside police had a “substantial chilling effect” on his right to free speech.

Miller, 54, from Lincolnshire, said an officer told him he had not committed a crime, but that his tweeting was being recorded as a “hate incident”.

In a strongly-worded judgement, Mr Justice Julian Knowles said the effect of police turning up at Miller’s place of work “because of his political opinions must not be underestimated”.

He said: “In this country we have never had a Cheka, a Gestapo or a Stasi. We have never lived in an Orwellian society,” he said.

But trans rights activists reacted with disappointment to parts of the judge’s ruling. Helen Belcher, co-founder of Trans Media Watch, said trans people would fear “open season” on them as a result. She added: “I think it will reinforce an opinion that courts don’t understand trans lives and aren’t there to protect trans people.”

Knowles rejected a wider challenge to the lawfulness of the College of Police guidance, ruling that it “serves legitimate purposes and is not disproportionate”. The guidance defines a transgender hate incident as “any non-crime incident which is perceived, by the victim or any other person, to be motivated by a hostility or prejudice against a person who is transgender or perceived to be transgender”.

Speaking outside the Royal Courts of Justice after the ruling, Miller called it “a watershed moment for liberty”.

Miller posted a number of tweets between November 2018 and January 2019 which he said formed part of the debate about proposed changes to the Gender Recognition Act 2004.

The tweets included: “I was assigned Mammal at Birth, but my orientation is Fish. Don’t mis species me.” Miller also tweeted: “Transwomen are women. Anyone know where this new biological classification was first proposed and adopted?”. He later wrote that the statement was “bollocks”.

Miller also retweeted a poem by someone else on Twitter which appeared to be directed to a trans woman and used the term “stupid man”, making a derisive reference to a trans woman’s genitalia.

On the same day, Stephanie Hayden, a trans woman, has won her case against a woman who called her a “pig in a wig” and made multiple Twitter accounts to send her anti-trans messages. The court found Kate Scottow guilty of persistently making use of a public communications network to cause annoyance/inconvenience and anxiety at St Albans magistrates court.

In the Miller case, the judge ruled that there was no evidence Miller’s tweets “were ‘designed’ to cause deep offence”, adding that: “The tweets were not directed at the transgender community. They were primarily directed at the claimant’s Twitter followers.” The tweets were lawful and there was not “the slightest risk” that he would commit a criminal offence by continuing to tweet, he ruled.

Knowles stressed “the vital importance of free speech”, saying it included “not only the inoffensive, but the irritating, the contentious, the eccentric, the heretical, the unwelcome and the provocative”.

In his judgment, Knowles said: “I conclude that the police’s actions led him, reasonably, to believe that he was being warned not to exercise his right to freedom of expression about transgender issues on pain of potential criminal prosecution.”

The judge also emphasised he was not “concerned with the merits of the transgender debate”, adding “[t]he issues are obviously complex”.

Deputy chief constable Bernie O’Reilly, executive director at the College of Policing, said he was pleased the court had recognised the guidance on recording non-crime hate incidents was “both lawful and extremely important in protecting people”.

He added: “Our guidance is about protecting people because of who they are and we know this is an area where people may be reluctant to report things to us.”

Cara English from the group Gendered Intelligence said trans people were facing “diabolical rhetoric”. She welcomed the conviction of Scorrow which she said would “help end harassment against trans people”.

Referring to the Miller case, she called for measured language. “If those users antagonistic to trans people freely living their lives could dial down the needlessly inciting language – particularly on Twitter – that may help to deescalate the unfortunate situation in which we find ourselves,” she said.

Kirrin Medcalf, head of trans inclusion at Stonewall, said Hate Crime Operational Guidance was “vital” because one in eight trans people had been assaulted at work. “Police must be able to record hate incidents so it’s important that this ruling re-affirmed that the guidance is lawful,” he said. “As a society, we must find ways to better support the two in five trans people who have faced a hate crime or incident in the last year.”

The judge granted Miller permission to appeal against his ruling on the lawfulness of the College of Police’s guidance at the supreme court.
I have written nothing on trans issues for seven years. A now-familiar row had broken out in the feminist movement back then, and I assumed that feminism would soon re-orient itself away from which body parts define a woman and whether or not the word “womxn” signified an assault on our sense of selves, and towards what I thought was obviously the more fundamental question of the movement: who has it worse? Feminism, in my life’s experience of it, takes the side of the oppressed. That is our raison d’etre.

So, anyway, I had seen this wonderful talk by Helen Belcher, who described the three ways in which trans people are portrayed and undermined, in the media and beyond. “The first is that they’re fraudulent. They’re not really who they say they are. We’d better humour them in their delusion. The second is trans as undeserving deviant. The third is trans as comedy.” Since then, this has intensified, with other, even more hostile, elements added: trans people as predators, the trans movement as deliberately poisoning the young. The savage mischief has seeped out of it. There is not much of the “We’d better humour them” any more.

Even in 2013, though, it was clear that trans people were in the eye of a familiar set of prejudices, which any of us – gay, female, disabled, BAME – might recognise. The difference was that people had become much less likely than they once were to laugh openly at those with disabilities, or to raise an eyebrow when two men kissed or an upstart woman demanded a fundamental human right. As a cause matures, it gets to the point where everyone recognises that their individual view no longer matters. But all that prejudice did not just evaporate, and the very idea of a trans person became its great release. All this pent up feeling exploded on to this one group, who – to put it mildly – could have done without it. It was, and remains, obvious which side feminism would be expected to take in this fight: the side of compassion and fellowship. We would recognise the importance of being an ally in a battle that we had been through.

Seven years ago, I thought this was quite an emollient argument, having gone nowhere near definitions, biology or absolutes – a call to rejoice in everything that made the women’s movement meaningful and victorious: strength in numbers, solidarity and, ultimately, love.

That is not how it played out. Women on the other side of this row were apoplectic, because if you celebrate compassion without explicitly agreeing with them, they would argue that you are calling them uncompassionate. I thought: I am never going anywhere near this again, which was cowardly; I regret it.

But I recall that not as a humble-brag-style apology, but because there is a bizarre idea ossifying that “real” feminists are being hounded out of the discursive space by trans activists. Rather, what has occurred is the systematic enclosure of the debate, so that unless you want to go to the mats about toilets, your point of view is not relevant.

All kinds of voices have been excluded. The experience of trans men, for instance, has been more or less erased, because the core issues have been whittled down to such a sharp, conflicted point – do cis women need protected status? – that the very existence of trans men has become too inconvenient to accommodate. The mainstream feminist view, which is trans-inclusive, has been sidelined to maintain the fiction that this is a generational battle between old and young feminists. Again, it is tactical and convenient to portray trans inclusion as a Trojan horse that all the young idiots allow in, being unaware of the history of women’s rights. But it simply is not so. Just because you are middle aged does not mean you agree with Germaine Greer. On the flipside, it is convenient to find that your voice is not relevant to this debate, because the conflict runs so high. But that is not a sufficient response when the generosity of the movement is at stake.

It is astonishing that the idea of the “women-only space” is being touted as a fundamental pillar of the movement, yet is completely stripped of the historical context of that. Women-only space was a realm protected from our Harvey Weinsteins, where we could talk about our Harvey Weinsteins; it was not a hallowed place where we communicated through our ovaries. It was where we came together in unity against people who hated us. I can’t imagine the mindset that would exclude a trans sister from that.

The fact is, in every backlash against every civil rights movement, there have been people saying: your emancipation is not possible for operational reasons. You can’t have the vote because your hands are too small. You can’t use the swimming pool because your hygiene is not the same . I find this egregious. What are we doing, trying to consecrate the public lavatory as a place so precious to the experience of womanhood that we have to be exclusive, rather than inclusive; that we have to characterise ourselves as a set of vulnerabilities, rather than strengths?

So much of the live combat happens on Twitter. This is not to call it irrelevant (I love Twitter), but might be cause to reflect. It really suits the “alt-right” to see feminist discourse mired in this. There is so much else we could be doing. The two major transnational grassroots movements of the recent past are the climate strikes and the women’s marches. Their combined energy – bearing in mind how much crossover there is in those communities – would be awesome.

We are witnessing an explosion of misogyny at the highest levels of public life. We have seen explicitly misogynistic terrorist acts – indeed, hatred of women is the through line that connects Isis to incels, that unites acts of violence globally – all while state-sponsored oppression of women has ascended to new levels of impunity. We have never needed unity more, yet, by wild coincidence, at exactly this moment, suddenly there is this obstacle that we can’t get past.

It feels distinctly modern and unprecedented that we find ourselves in this obliterative debate, where one side can’t prevail until the other is destroyed, where all the values of the movement are parked so we can fight over issues that are so technical that there is no room for compromise, yet also so abstract that there is no space for human beings.

In fact, there is precedent: we have weathered absolutes before. Are all men rapists? (But what about my son?) Is all sex servitude? (What if it was my idea?) Is lesbianism a political act? (What if I hate politics and would prefer to have a sexual destiny like anyone else?) None of this was settled in science. There was no definitive treaty that we all signed. We just decided slowly that binary debates are interesting to have, but boring to live. Solidarity is boring to talk about, but fascinating and empowering to live. Solidarity is not exclusive or pedantic; it is compassionate and fights oppression where it finds it. That is its lifeblood. That is why trans women are women, or womxn.
A transgender man in Georgia is appealing against a judge’s refusal to grant his legal name change.

Columbia County superior court judge J David Roper in March rejected a name change petition from the transgender man seeking to legally change his name from Rebeccah Elizabeth Feldhaus to Rowan Elijah Feldhaus. LGBT rights group Lambda Legal on Thursday submitted a filing to the Georgia court of appeals challenging the denial.

“The question presented is whether a female has the salutatory right to change her name to a traditionally and obviously male name,” the judge wrote. “The court concludes that she does not have such right.”

Feldhaus, 24, is a sergeant in the US army reserve and works in guest services at an Augusta-area hotel. He has been diagnosed with and is being treated for gender dysphoria, which is characterized by stress stemming from conflict between one’s gender identity and assigned sex at birth, the appeal says. He is receiving hormone treatments, lives as a man, and his friends, family and co-workers call him Rowan.

He never felt comfortable with his given name, mostly because he didn’t feel like he was in the body he was supposed to be in, he said.

About a year and a half ago, one of his best friends, who was a bit tipsy at the time, said that if Feldhaus was a guy, he could see him as a Rowan. It was right at the moment when Feldhaus was coming out to friends and family and struggling to find a name that felt comfortable and that name just clicked, he told the Associated Press by phone.

“It was a very grounding moment,” he said.

He chose Elijah as a middle name because it sounds similar to his given middle name.

The judge’s decision made him feel “insulted and objectified”, he said.

Roper said he does not approve of changing a name for someone who is anatomically one sex to a name that is obviously used for the opposite sex. He said it can be misleading for the public and can be dangerous if people don’t know whether they’re dealing with a man or a woman, according to a transcript of a February hearing on Feldhaus’s petition.

But Roper said he recognized that Feldhaus is not going to stop presenting himself as male and that this could present problems for people who have to interact with him. For that reason, despite his own disapproval, he would allow a change to a gender-neutral name, he said.

“I will allow a gender-neutral name change that will benefit the general public because I don’t want them to have to go through the embarrassing issue of trying to figure out what to do with you when you present, in your appearance today, with a female name, particularly if you had on a uniform and you were dressed like a man,” Roper said, according to the transcript.

For that reason, Roper said he would approve a change to Rowan because it is sufficiently gender neutral, but he rejected Feldhaus’s chosen middle name, Elijah, because it is clearly a man’s name.

The Georgia law governing name changes does not deal with transgender name changes and there is no appellate court decision on the issue, Roper wrote in his order denying the change. But he noted that the law does not allow a name change “with a view to deprive another fraudulently of any right under the law”.

“Name changes which allow a person to assume the role of a person of the opposite sex are, in effect, a type of fraud on the general public,” Roper wrote. “Such name changes also offend the sensibilities and mores of a substantial portion of the citizens of this state.”

Feldhaus’s attorneys argue that Roper abused his discretion when he denied the name change petition because the denial was arbitrary and based on insufficient and improper reasons rather than being based on evidence of fraud or improper motive. The legal discretion accorded to judges is not an invitation to rule based on private opinions, the appeal says. The lawyers say the judge’s denial constitutes sex discrimination and a violation of the petitioner’s constitutional rights.

Feldhaus had also submitted a statement to the judge from his therapist saying that changing his name is “crucial” for his course of treatment.
The Boy Scouts of America now allows transgender children who identify as boys to enroll in its boys-only programs.

The organization said on Monday it had decided to begin basing enrollment in its boys-only programs on the gender a child or parent lists on the application to become a scour, rather than birth certificate.

Rebecca Rausch, a spokeswoman for the organization, said the organization’s leadership had considered a recent case in Secaucus, New Jersey, where an eight-year-old transgender child had been asked to leave his Scout troop after parents and leaders found out he is transgender, but that the change was made because of the national conversation about gender identity.

“For more than 100 years, the Boy Scouts of America, along with schools, youth sports and other youth organizations, have ultimately deferred to the information on an individual’s birth certificate to determine eligibility for our single-gender programs,” the statement said.

“However, that approach is no longer sufficient as communities and state laws are interpreting gender identity differently, and these laws vary widely from state to state.”

Rausch said the enrollment decision went into effect immediately.

“Our organization’s local councils will help find units that can provide for the best interest of the child,” the statement said.

Boy Scouts of America leaders lifted a blanket ban on gay troop leaders and employees in July 2015.
